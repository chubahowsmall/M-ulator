\documentclass{article}
%\usepackage{fullpage}
\usepackage[top=1in, bottom=1in, left=1cm,right=1cm]{geometry}
\usepackage{hyperref}
\usepackage{fancyhdr}

\begin{document}

\def\checkptdate{Friday, Sep 23, 10:30 AM}
\def\duedate{Friday, Sep 30, 10:30 AM}

\fancyfoot[L]{Revision 1.3 -- 9/22/2011}
\pagestyle{fancyplain}

\title{EECS 373 Cortex M3 Simulator}
\author{{\em Questions? Issues?} E-Mail {\tt 373F11class@umich.edu} with [proj1]
in Subject}
\date{Phase 1 -- Assigned: Sep 15, {\bf Due: \duedate}}
\maketitle

The goal of this project is to teach everyone a little about their little
corner of the Cortex M3, as well as a little bit about the challenges of
building and designing a system where everyone needs to work together, and
everyone's work depends on everyone else.

You should read this whole document before trying anything else, it will save
you a lot of pain (and isn't that long -- promise).

\section{General Project Information}

This project is likely unlike any project you have ever done for an EECS class
before. While each of you have your own tasks you are required to implement,
the entire class is working on the {\em same} project. This has some very
interesting dependency implications; notably that everyone is depending on
everyone else. If you write the perfect {\tt mov} instruction, but the person
responsible for the {\tt add} or {\tt sub} instructions hasn't written them
yet, the simulator probably won't get very far.
In addition, this model will make ``debugging by trial'' very hard. You will
have to think hard about every line of code you write to ensure it is correct.

\subsection{Subversion}

To collaborate, we will be using subversion. If you have never used subversion
before, take a few minutes to check out this primer at {\tt
\href{http://www.eecs.umich.edu/~pmchen/eecs482/svn.pdf}
{www.eecs.umich.edu/\textasciitilde pmchen/eecs482/svn.pdf}}. The repository is
located at

\begin{quote}
{\tt svn co
svn+ssh://UNIQNAME@loginlinux.engin.umich.edu/afs/umich.edu/user/p/p/ppannuto/Public/373F11~proj1}
\end{quote}

and should be accessible
from any CAEN machine or via ssh to {\tt loginlinux.engin.umich.edu}.

It is in everyone's best interest to make commit messages with useful
information -- particularly since they will be relied heavily upon for
grading. The first line of your commit message should be a short (60ish
characters) summary of the commit.
Every time someone makes a commit, an email
will be sent to {\tt 373F11class@umich.edu} with subject {\tt [svn] --
First-line-of-the-commit-message}.

A quick reminder, recall that everyone is sharing this repository, so try to
keep expletives or other such frustrations to a minimum. Also, don't clutter
the repository by adding .o or other compiled files to the repository. Stub .c
files are supplied for you already.

\subsection{Grading}

Since everyone is working on the same project, we will be relying on
subversion to attribute credit for work done on the project. There is not a
fixed grading rubric, as the requirements vary somewhat depending what portion
of the simulator you have been assigned. The expectation is that the simulator
will reach a point where it is working 100\% accurately. We will be actively
monitoring the class's progress to help ensure we meet that end. Please take
careful note of the deadlines; if you slip a deadline, we will write your
portion of the project for you, and if that happens you should expect your
grade to drop drastically.

\subsection{Deadlines}

The {\em entire} simulator is due \duedate. We will have one checkpoint, due
\checkptdate. The purpose of the checkpoint is to ensure that a few
procrastinators do not cause the entire class to fall behind on the project.
By the checkpoint you are required to have made a serious, appreciable effort
towards the completion of your portion of the project. This is a subjective
measure, and we will send warning emails out 24 hours before the deadline if
it looks like you aren't going to make it. You are very strongly advised,
however, that a haphazard commit Friday morning most definitely does not
qualify.

\section{Building the Simulator}

This section deals with code in the root directory and the {\tt operations/}
directory.

\subsection{Simulator Architecture}

You should be familiar with the Cortex M3 from lecture and other resources.
This section explains the architecture of the M3 simulator. In general,
everyone has one or two functions they will have to fill in the body for. In
writing your functions, you will have to use accessor functions. These are all
sorted and prototyped in {\tt cortex\_m3.h}.

\begin{quote}
  \emph{YOU SHOULD NOT HAVE TO MODIFY THE cortex\_m3.h FILE}
\end{quote}

Our simulated chip has Power-On Reset, so the first function called will be
the {\tt reset} function. Recall we are simulating the processor, so this
function must do everything a real Cortex M3 would do when the reset pin is
asserted.

You will notice there is no {\tt main()} function anywhere - the simulator
program entry (main function) is in the provided {\tt simulator.o} file. The
simulator architecture will call your functions as necessary to correctly
simulate a Cortex M3. To draw example from the previous paragraph, when the
simulator begins running, the first function it will call is {\tt reset()}.

Every cycle the simulator will (like a real processor) read the value of the
{\tt PC} register and fetch that instruction. The simulator will then check if
it knows how to execute that instruction and either proceed or quit.

\subsection{Registering Opcodes, or ``Teaching the Simulator''}

Say for example the simulator fetches the instruction {\tt 0xb083}. (This is
{\tt sub sp, \#12}). The simulator will scan the list of all registered opcode
masks to see if it knows how to execute this instruction. A pair of opcode
masks are made up of the ``{\tt ones\_mask}'' and ``{\tt zeros\_mask}''. These
masks represent the bits that {\em must} be one and {\em must} be zero
respectively. For example the masks {\tt (0xb0c2,0x4f40)} would match {\tt
0xb083} while {\tt (0xb080,0x4f80)} would not. It is worth noting that some
instructions may require multiple masks.

\begin{quote}
  {\it Exercise: Why do you need two masks?}
\end{quote}

To register an opcode mask is to ``claim'' that instruction as one that you
know how to execute. The implementation of the {\tt sub} instruction is then
done something like this:

\begin{verbatim}
operations/sub.c:

#include "../cortex_m3.h"

int sub4(uint32_t inst) {
        uint32_t new_sp_val;
        // ...
        CORE_reg_write(SP_REG, new_sp_val);
        return SUCCESS;
}

void register_opcodes_sub(void) {
        // This instruction is a 16-bit instruction, so we require
        // the top halfword of the instruction to be all 0's
        register_opcode_mask(0xb0c2, 0xffff4f40, sub4);
}
\end{verbatim}

n.b.: The given examples are {\bf not} the correct masks for sub4

\vspace{5mm}

The implementation for all of the Thumb instructions can be found in the
operations directory. Since this is a shared project, the implementation is
broken out into different files, where the filenames are taken from the boxes
of the ARM Quick Reference Card.

Please be sure to implement functions in the correct files, if in doubt --
please ask (or consult {\tt operations/MISSING.txt}).


\subsection{Running The Simulator}

The supplied Makefile will generate an executable called simulator. The
simulator does not require any arguments, but has many options that should
make testing and debugging at least a little less painful ({\tt ./simulator
--help}).

To compile the simulator, simply type ``{\tt make}'' in the root directory.
You may alternatively choose ``{\tt make debug=1}'' or ``{\tt make debug=2}'' to
enable debugging output. In general, try to keep ``Level 1'' debugging
statements to things that do not print often (i.e. won't flood the screen when
run). See {\tt cortex\_m3.h} for more details on the debugging controls.

To run the simulator, simply execute {\tt ./simulator}. The simulator will
attempt to load the file {\tt flash.mem} into ROM and then begin executing.

Until you are capable of generating valid programs, you can use the {\tt
--usetestflash} flag to use an internal copy of the supplied {\tt echo}
program.

\subsection{Simulator Peripherals}

Our simulator has two peripherals, some LEDs and a UART. For now, these
peripherals are simply registers in memory. Documentation on reading and
writing these devices can be found in {\tt memmap.txt}.

\subsubsection{LEDs}

When an LED's status is changed, a line will print to the screen showing the
current state of all LEDs.

\subsubsection{UARTs}

To communicate with the processor's UART we will use the {\tt netcat} program
as the other peripheral. Once the simulator is running, in another terminal
(on the same machine -- careful if you're using SSH) run {\tt nc -4 localhost
4100} to send and receive bytes.

\section{Programming The Simulator}

This section deals with code in the {\tt programs/} directory.

\subsection{Peripheral Support Libraries}

Each of the peripherals (LEDs and UART) have support libraries found in {\tt
programs/lib/}. The files {\tt led.h} and {\tt uart.h} are supplied for you,
but you are responsible for implementing {\tt led.s} and {\tt uart.c}.

To make the LED peripheral more challenging, you are required to write the
function {\tt LED\_set} {\em in assembly}. In addition, your {\tt LED\_set}
implementation cannot use any branch instructions (except bx to return).
Lastly, you may only use 16-bit instructions for this section. This may seem
challenging at first, but once you ``get it'' you'll really be thinking ARM!

Code for the UART peripheral should be written in C.

\subsection{vector.s and memmap}

These files are both located in the {\tt programs/} directory, and will need
to be completed before you can generate code.

Once these files are complete, you should be able to type {\tt make} in the
{\tt programs/} directory and compile our test programs.

\subsection{Test Programs}

The simulator should run any valid thumb2 assembly. We have supplied three
test programs, {\tt blink.c}, {\tt echo.c}, and {\tt echo\_str.c}. In addition
to these programs you are required to write test programs to verify the
validity of the simulator.

Your test programs should exit with return code 0 (that is from {\tt main} it
should {\tt return 0;}) if the test passes or any non-zero return code. Your
tests must contain legal ARM code and should not throw any exceptions.

\section{General Hints}

\subsection{First Steps}
\label{sec:firststeps}

\begin{enumerate}
\item Install the ARM toolchain (Confused? See \ref{sec:missing})
\item Enter the {\tt programs} directory and type {\tt make} to build a copy
of all the programs designed to run on the simulator. Only the {\tt basic}
program will work at first
\subitem The {\tt Makefile} generated four different file types: .o .elf .list
.bin.  The .bin file is a binary image, something you could flash onto real
hardware, or load into the simulator via {\tt --flash}. What are the other
files?
\item Back in the root directory, run {\tt ./reference --flash
programs/basic.bin --dumpallcycles}. What happened?
\item Now try {\tt arm-[...]-eabi-objdump -Sd programs/basic.elf}. What is
this output? How does it relate to what happened in the previous step?
\item Type {\tt make} in the root directory to build your own copy of the
simulator. Try the same command {\tt ./simulator --flash programs/basic.bin
--dumpallcycles}. What happened this time?
\item You've got the basics! Now start reading through the code and try to
figure out what's going on.
\end{enumerate}

\subsection{Debugging}
\label{sec:debugging}

Debugging this project will be challenging (so start early!). Check out all
the options you get with {\tt ./simulator --help}. Try to eliminate as many
variables as you can. If you're trying to debug the simulator, take advantage
of the {\tt --usetestflash} option so you know you're running ``good'' code.
On the flip side if you're debugging programs or libraries you've written, use
the {\tt reference} simulator to execute it.

\subsection{TTTA}
\label{sec:TTTA}
Sprinkled throughout the code are ``{\tt TTTA}'' comments (Things To Think
About).  These will either help with your implementation of the simulator or
your understanding of the M3 - or both! Be sure you can answer all the
{\tt TTTA}'s by the time you finish the project... (even, or especially, those
from other people's sections)

\vspace{5mm}

{\em Hint: {\tt grep -nR TTTA *.s *.c *.h} to find all of them}

\subsection{Missing Information?}
\label{sec:missing}
At times, you may feel like you don't have enough information. {\em You
probably don't}. A key part of this project is learning how to find resources
and how to extract the information you need from them. As an example, missing
from this document is exactly how to decode instructions (such as {\tt 0xb083}
to {\tt sub sp, \#12}).

Use the resources available to you to try to figure out missing information.
Also do not hesitate to use the mailing list, both to ask questions and to
share resources you think others may find useful.

\section{One Last Thought...}
This project is a bit of an experiment. We hope things go well, but please be
understanding if things are a bit fluid. To that end, email frequently, ask
questions!

\pagebreak

\section{Document Revision History}

\begin{itemize}

\item Revision 1.3b
\subitem Remove LED printing option, always on

\item Revision 1.3a
\subitem showledtoggles --> showledwrites

\item Revision 1.3
\subitem Clarified due dates mismatch

\item Revision 1.2
\subitem Added section \ref{sec:firststeps}: First Steps
\subitem Added section \ref{sec:debugging}: Debugging

\item Revision 1.1
\subitem Add svn email notice
\subitem Added explicit 16-bit instruction requirement to led.s
\subitem Dropped the 16-bit instruction only limitation
\subitem Clarified svn checkout
\subitem Cleaned up document spacing

\item Revision 1.0a
\subitem Fixed typo in svn+ssh command

\item Revision 1.0
\subitem Initial Release

\end{itemize}

\end{document}
