\subsection{Version 0.2}
\label{protocol-0-2}

Version 0.2 adds the following new messages:
\begin{itemize}
  \item {\tt `Oo'} and {\tt `oo'}. These messages allow the default on/off state of
the GOC light to be set.
  \item {\tt `??'}. This message queries the capabilities of the ICE board.
  \item {\tt `?B'}, {\tt `?b'}. These messages get and set the ICE baud rate.
  \item {\tt `B'}, {\tt `b'}, {\tt `M'}, and {\tt `m'}. These messages control
    the MBus interface.
  \item {\tt `e'}. This command injects messages on the {\tt EIN} debug port.
\end{itemize}

\setcounter{tocdepth}{4}
\etocsettocstyle
    {\subsection*{\contentsname}\hrule\medskip
        \everypar{\rightskip.1\linewidth}}
    {\nobreak\medskip\hrule\bigskip}

\localtableofcontents

\subsubsection{\texttt{0x3f `?'} -- Query ICE}
{\em Synchronous Request}
\begin{itemize}
  \item The first byte of this message shall be a specificer, indicating
    what type of query is requested.
  \paragraph{\texttt{0x3f 0x3f `??'} -- Query capabilities}
    \begin{itemize}
      \item This message shall query the capabilities of this ICE board. Not
        all boards have every hardware frontend. Use this message to query the
        capabilities of a given ICE board.
      \item[]
        \begin{bytefield} \\
          \colorbitbox{lightgreen}{8}{`?' (0x3f)} &
          \colorbitbox{lightred}{8}{Event ID} &
          \colorbitbox{lightcyan}{8}{Len (Must be 1)} &
          \colorbitbox{lightergreen}{8}{`?' (0x3f)}
        \end{bytefield}
      \item This message shall be responded to with a list of all top-level
        identifiers that the ICE board is capable of acting usefully upon.
        That is, if the ICE firmware understands {\tt `d'} messages but does
        not have an I2C frontend {\tt `d'} shall be omitted. Sub-types are not
        specified by this command. That is, if a version 0.2 ICE includes
        {\tt `o'} in its response it is assumed to understand both the
        {\tt `oc'} and {\tt `oo'} messages.
      \item Both set and query commands should be included in this list. That
        is, if a GOC frontend is present, both {\tt `o'} and {\tt `O'}
        should be included.
      \item As example, a version 0.2 ICE with no physical I2C frontend would
        respond:
      \item[]
        \begin{bytefield} \\
          \colorbitbox{lightgreen}{8}{ACK (0x00)} &
          \colorbitbox{lightred}{8}{Event ID} &
          \colorbitbox{lightcyan}{8}{Len (11)} &
          \bitbox{20}{\texttt{`?\_IifOoGgPp' (0x3f5f4969664f6f47675070)}} &
        \end{bytefield}
    \end{itemize}
  \paragraph{\texttt{0x3f 0x62 `?b'} -- Query baudrate}
    \begin{itemize}
      \item This message shall mirror the {\tt `\_b'} message and report the
        currently set baud rate.
    \end{itemize}
  \paragraph{\texttt{0x3f} Responses}
    \begin{itemize}
      \item NAKs for this message shall be composed of an error
        code, followed by an optional explanatory string.
        \begin{itemize}
          \item[]
            \begin{bytefield} \\
              \colorbitbox{lightgreen}{8}{NAK (0x01)} &
              \colorbitbox{lightred}{8}{Event ID} &
              \colorbitbox{lightcyan}{8}{Len (Min: 1)} &
              \colorbitbox{lightblue}{8}{EINVAL (0x16)} &
              \bitbox{8}{[{\tt "Out of Range"}]} &
            \end{bytefield}
          \item {\texttt {\textbf EINVAL (22,0x16):}} Invalid argument.
        \end{itemize}
    \end{itemize}
\end{itemize}

\subsubsection{\texttt{0x5f `\_'} -- Configure ICE}
{\em Synchronous Request}
\begin{itemize}
  \item The first byte of this message shall be a specificer, indicating
    what type of query is requested.
  \item {\tt `\_'} commands are setters for the {\tt '?'} getters if
    appropriate.
  \paragraph{\texttt{0x5f 0x62 `\_b'} -- Set baudrate}
    \begin{itemize}
        \item[]
          \begin{bytefield} \\
            \colorbitbox{lightgreen}{8}{`\_'} &
            \colorbitbox{lightred}{8}{Event ID} &
            \colorbitbox{lightcyan}{8}{Len (Must be 3)} &
            \colorbitbox{lightergreen}{8}{`b'} &
            \bitbox{8}{Clock Divider}
          \end{bytefield}
      \item This message shall set the baud rate for future messages. The new
        baud rate shall take effect after the ACK for this
        request.
      \item This shall be followed by a two byte
        value N (MSB-first), where $N = \lceil\frac{20~MHz}{Baud Rate}\rceil$.
      \item {\bf Default:} {\tt 0x00AE} (20~MHz / 0x00AE $\approx$ 115200~Hz)
      \item The following message would set the ICE speed to 3 Megabaud:
      \item[]
        \begin{bytefield} \\
          \colorbitbox{lightgreen}{8}{`\_' (0x5f)} &
          \colorbitbox{lightred}{8}{Event ID} &
          \colorbitbox{lightcyan}{8}{Length (3)} &
          \colorbitbox{lightergreen}{8}{`b' (0x62)}
          \bitbox{8}{0x0007}
        \end{bytefield}
    \end{itemize}
\end{itemize}

\subsubsection{\texttt{0x64 `d'} -- Discrete interface I2C message}
{\em Synchronous Request, Asynchronous Message}
\begin{itemize}
  \item[]
    \begin{bytefield}{40}
      \colorbitbox{lightgreen}{8}{`d'} &
      \colorbitbox{lightred}{8}{Event ID} &
      \colorbitbox{lightcyan}{8}{Length} &
      \bitbox{4}{Data...}
    \end{bytefield}
  \item The bytes in this message compose an I2C transaction.
  \item The length field of this message necessarily limits the maximum
    message size to 255 bytes (addr + 254 data). Longer messages should
    be fragmented.
  \item The sentinel value {\tt 255} for length indicates a {\em
    fragmented} message.
  \item[]
    \begin{itemize}
      \item A message fragment {\bf MUST} be exactly 255 bytes long.
      \item A fragment message {\bf MUST} be followed by another
        fragment, or terminated by a regular {\bf d}iscrete message.
        \begin{itemize}
          \item The ICE board is permitted to interleave other,
            non-I2C related messages (e.g. GPIO events).
        \end{itemize}
      \item A series of message fragments {\bf MUST} always be
        terminated by a discrete with an explicit length. An I2C transaction of
        length 0 is permitted, e.g.:
        \begin{itemize}
          \item An I2C transaction of length 510 (1 byte addr + 509 bytes
            of data) would be {\bf three} messages. The first of length
            255 (addr + bytes 0-253), the second of length 255 (bytes
            254-508), and the third of length 0 (there is no more
            data, but the fragment series must be terminated).
        \end{itemize}
      \item Fragments are treated as one logical message, but individual I2C bus
        transactions, by an ICE board.  In practice this means:
        \begin{itemize}
          \item Each fragment message must be individually ACK'd by ICE.
            \begin{itemize}
              \item A NAK'd fragment message ends an I2C message.
              \item The NAK offset is relative to the current fragment, not
                the whole I2C transaction.
            \end{itemize}
          \item Only the first fragment includes the I2C address.
          \item A stop bit should {\bf NOT} be generated after a
            fragment, instead the I2C clock should be stretched until the
            next fragment has arrived.
        \end{itemize}
    \end{itemize}
  \item ICE will respond with an ACK once every byte from an individual `d'
    message has been ACK'd on the I2C bus.
  \item If a byte is NAK'd on the I2C bus, ICE will respond with a NAK message
    of length 1 indicating the index of the first NAK'd byte (e.g. if the
    address is NAK'd, it will return 0).
    \begin{itemize}
      \item[]
        \begin{bytefield}{32}
          \colorbitbox{lightgreen}{8}{NAK (0x01)} &
          \colorbitbox{lightred}{8}{Event ID} &
          \colorbitbox{lightcyan}{8}{Len (Must be 1)} &
          \bitbox{8}{Index of Byte NAK'd}
        \end{bytefield}
    \end{itemize}
\end{itemize}

\subsubsection{\texttt{0x49 `I'} -- Query I2C Configuration}
{\em Synchronous Request}
\begin{itemize}
  \item[]
    \begin{bytefield} \\
      \colorbitbox{lightgreen}{8}{0x49} &
      \colorbitbox{lightred}{8}{Event ID} &
      \colorbitbox{lightcyan}{8}{Length} &
      \bitbox{8}{Parameter} &
    \end{bytefield}
  \item These messages complement the set I2C messages.
  \item The `Parameter' field is the parameter specifier to query.
  \item An ACK response should mimic the corresponding set message.
  \item The following would query/response the address mask:
    \begin{quote}
      \begin{bytefield} \\
        \colorbitbox{lightgreen}{8}{0x49} &
        \colorbitbox{lightred}{8}{Event ID} &
        \colorbitbox{lightcyan}{8}{0x01} &
        \colorbitbox{lightergreen}{8}{0x61} &
      \end{bytefield}

      \begin{bytefield} \\
        \colorbitbox{lightgreen}{8}{ACK (0x00)} &
        \colorbitbox{lightred}{8}{Event ID} &
        \colorbitbox{lightcyan}{8}{0x02} &
        \bitbox{8}{Ones Mask} &
        \bitbox{8}{Zeros Mask} &
      \end{bytefield}
    \end{quote}
\end{itemize}

\subsubsection{\texttt{0x69 `i'} -- Set I2C Configuration}
{\em Synchronous Request}
\begin{itemize}
  \item The first byte of the message shall define which parameter is to
    be configured.

  \paragraph{\texttt{0x69 0x63 `ic'} -- Set I2C Clock Speed}
    \begin{itemize}
      \item[]
        \begin{bytefield} \\
          \colorbitbox{lightgreen}{8}{`i'} &
          \colorbitbox{lightred}{8}{Event ID} &
          \colorbitbox{lightcyan}{8}{Len (Must be 2)} &
          \colorbitbox{lightergreen}{8}{`c'} &
          \bitbox{8}{Clock Speed}
        \end{bytefield}
      \item {\bf Default:} {\tt 0x32} (50, 100~kHz)
      \item This shall be followed by a single byte
        valued N, where N~*~2~kHz yeilds the desired clock speed. Values of N
        greater than 200 (400~kHz) exceed the I2C spec and may be rejected.
    \end{itemize}
  \paragraph{\texttt{0x69 0x61 `ia'} -- Set ICE I2C Address}
    \begin{itemize}
      \item[]
        \begin{bytefield} \\
          \colorbitbox{lightgreen}{8}{`i'} &
          \colorbitbox{lightred}{8}{Event ID} &
          \colorbitbox{lightcyan}{8}{Len (Must be 3)} &
          \colorbitbox{lightergreen}{8}{`a'} &
          \bitbox{8}{Ones Mask} &
          \bitbox{8}{Zeros Mask}
        \end{bytefield}
      \item {\bf Default:} {\tt 0xff 0xff} (disabled)
      \item This shall be followed by two bytes,
        first the {\em ones mask} and then the {\em zeroes mask} as outlined below.
        The command sets the address mask that ICE board should pretend to be a device
        for. Coneceptually the mask is of the form {\tt 10xx010x}, where x's signify
        don't care. This is conveyed as a {\em ones mask} and a {\em zeroes mask},
        where each mask defines the bits that must be a one or zero respectively. For
        the given example, the ones mask would be {\tt 10000100} and the zeroes mask
        {\tt 01001010}, generating a transaction of \mbox{\tt 0x61 0x84 0x4a}.
      \item To disable address-faking, set any bit
        as both required-one and required-zero. This impossible situation is a legal
        setting that will never match.
        \begin{itemize}
          \item {\em Note:} While it is
            permissable to set the last bit must-be-zero (writeable-only) or must-be-one
            (readable-only), it is almost certainly an error to do so.
        \end{itemize}
    \end{itemize}
  \paragraph{\texttt{0x69} Responses}
    \begin{itemize}
      \item NAKs for this message shall be composed of an error
        code, followed by an optional explanitory string.
        \begin{itemize}
          \item[]
            \begin{bytefield} \\
              \colorbitbox{lightgreen}{8}{NAK (0x01)} &
              \colorbitbox{lightred}{8}{Event ID} &
              \colorbitbox{lightcyan}{8}{Len (Min: 1)} &
              \colorbitbox{lightblue}{8}{EINVAL (0x16)} &
              \bitbox{8}{[{\tt "Out of Range"}]} &
            \end{bytefield}
          \item {\texttt {\textbf EINVAL (22,0x16):}} Invalid argument.
          \item[]
            \begin{bytefield} \\
              \colorbitbox{lightgreen}{8}{NAK (0x01)} &
              \colorbitbox{lightred}{8}{Event ID} &
              \colorbitbox{lightcyan}{8}{Len (Min: 1)} &
              \colorbitbox{lightblue}{8}{ENODEV (0x13)} &
            \end{bytefield}
          \item {\texttt {\textbf ENODEV (19,0x13):}} The
            implementation does not support changing or querying this parameter. Unless
            otherwise specified, it {\bf MUST} be hardcoded to the default.
        \end{itemize}
    \end{itemize}
\end{itemize}

\subsubsection{\texttt{0x66 `f'} -- Debug (GOC/EIN) interface message}
{\em Synchronous Request}
\begin{itemize}
  \item Debug interface messages are formatted the exact same as `d'iscrete messages.
\end{itemize}

\subsubsection{\texttt{0x4f `O'} -- Query debug (GOC/EIN) Configuration}
{\em Synchronous Request}
\begin{itemize}
  \item[]
    \begin{bytefield} \\
      \colorbitbox{lightgreen}{8}{`O' (0x4f)} &
      \colorbitbox{lightred}{8}{Event ID} &
      \colorbitbox{lightcyan}{8}{Len (Must be 1)} &
      \bitbox{8}{Parameter} &
    \end{bytefield}
  \item These messages complement the set {\tt `o'} messages.
  \item The `Parameter' field is the parameter specifier to query.
  \item An ACK response should mimic the corresponding set message.
\end{itemize}

\subsubsection{\texttt{0x6f `o'} -- Set debug (GOC/EIN) Configuration}
{\em Synchronous Request}
\begin{itemize}
  \item The first byte of the message shall define which parameter is to
    be configured.
    \paragraph{\texttt{0x6f 0x63 `oc'}: Clock Speed (Divider)}
      \begin{itemize}
        \item[]
          \begin{bytefield} \\
            \colorbitbox{lightgreen}{8}{`o'} &
            \colorbitbox{lightred}{8}{Event ID} &
            \colorbitbox{lightcyan}{8}{Len (Must be 4)} &
            \colorbitbox{lightergreen}{8}{`c'} &
            \bitbox{8}{Clock Divider}
          \end{bytefield}
        \item {\bf Default:} {\tt 0x30D400} (2~MHz / 0x30D400 = .625~Hz)
        \item This shall be followed by a three byte
          value N (MSB-first), where 2~MHz / N yields the desired clock speed.
      \end{itemize}
    \paragraph{\texttt{0x6f 0x70 `op'}: Debug mode}
      \begin{itemize}
        \item[]
          \begin{bytefield} \\
            \colorbitbox{lightgreen}{8}{`o'} &
            \colorbitbox{lightred}{8}{Event ID} &
            \colorbitbox{lightcyan}{8}{Len (Must be 2)} &
            \colorbitbox{lightergreen}{8}{`p'} &
            \bitbox{8}{GOC / EIN}
          \end{bytefield}
        \item {\bf Default:} {\tt 0x0} (EIN)
        \item Specifies which interface to use when transferring debug data.  Possible values:
          \begin{itemize}
            \item[0] EIN
            \item[1] GOC
          \end{itemize}
      \end{itemize}
    \paragraph{\texttt{0x6f 0x6f `oo'}: GOC light On/Off}
      \begin{itemize}
        \item[]
          \begin{bytefield} \\
            \colorbitbox{lightgreen}{8}{`o'} &
            \colorbitbox{lightred}{8}{Event ID} &
            \colorbitbox{lightcyan}{8}{Len (Must be 2)} &
            \colorbitbox{lightergreen}{8}{`o'} &
            \bitbox{8}{On / Off}
          \end{bytefield}
        \item {\bf Default:} {\tt 0x0} (Off)
        \item This shall be followed by either a {\tt 0} or a {\tt 1}
          indicating whether the default state of the light should be on or
          off. If the default is set to on, the light shall remain illuminated
          until set to off or a GOC pulse temporarily turns if off.
      \end{itemize}
  \paragraph{\texttt{0x6f} Responses}
    \begin{itemize}
      \item NAKs for this message shall be composed of an error
        code, followed by an optional explanatory string.
        \begin{itemize}
          \item[]
            \begin{bytefield} \\
              \colorbitbox{lightgreen}{8}{NAK (0x01)} &
              \colorbitbox{lightred}{8}{Event ID} &
              \colorbitbox{lightcyan}{8}{Len (Min: 1)} &
              \colorbitbox{lightblue}{8}{EINVAL (0x16)} &
              \bitbox{8}{[{\tt "Out of Range"}]} &
            \end{bytefield}
          \item {\texttt {\textbf EINVAL (22,0x16):}} Invalid argument.
        \end{itemize}
    \end{itemize}
\end{itemize}


\subsubsection{\texttt{0x42 `B'} -- MBus Snooped Message}
{\em Asynchronous Message}
\begin{itemize}
  \item[]
    \begin{bytefield} \\
      \colorbitbox{lightgreen}{8}{`B'} &
      \colorbitbox{lightred}{8}{Event ID} &
      \colorbitbox{lightcyan}{8}{Len (Min: 5)} &
      \colorbitbox{lightergreen}{8}{MBus Address} &
      \bitbox{8}{MBus Data} &
      \bitbox{8}{MBus Control}
    \end{bytefield}
  \item This shall be followed by a 32-bit MBus destination address as well as optional data.
    The current MBus implementation operates on 4-byte words, so the data field length must be a multiple of four.
    As with discrete ({\tt `d'}) messages, the length field limits the maximum message size to 255 bytes.  Longer messages should be fragmented.
  \item At the end of the message, snoop messages shall append one additional
    byte. This additional byte shall indicate the control bits of the snooped
    message. Control Bit~0 shall be mapped to bit 0 and Control
    Bit~1 shall be mapped to bit 1. The remaining bits are undefined.
  \item This message is sent only from the ICE board to report messages it has
    snooped (but not ACK'd).
  \item Messages that match both the ICE address and the snoop address (that
    is, would generate both {\tt `b'} and {\tt `B'} messages) shall not report
    snoop messages.
  \item The ICE board shall not report snoop messages for message that it is sending.
\end{itemize}

\subsubsection{\texttt{0x62 `b'} -- MBus Message}
{\em Synchronous Request, Asynchronous Message}
\begin{itemize}
  \item MBus messages are formatted the exact same as `MBus Snooped' ({\tt `B'}) messages.
  \item This command may be sent to the ICE board to send a message.
  \item This message is sent from the ICE board to report messages sent to
    ICE. That is messages that matched the address masks assigned to ICE and
    that ICE has ACK'd on the bus.
\end{itemize}

\subsubsection{\texttt{0x4d `M'} -- Query MBus Configuration}
{\em Synchronous Request}
\begin{itemize}
  \item[]
    \begin{bytefield} \\
      \colorbitbox{lightgreen}{8}{`M' (0x4d)} &
      \colorbitbox{lightred}{8}{Event ID} &
      \colorbitbox{lightcyan}{8}{Length} &
      \bitbox{8}{Parameter} &
    \end{bytefield}
  \item These messages complement the set of {\tt `m'} messages.
  \item The `Parameter' field is the parameter specifier to query.
  \item An ACK response should mimic the corresponding set message.
\end{itemize}

\subsubsection{\texttt{0x6d `m'} -- Set MBus Configuration}
{\em Synchronous Request}
\begin{itemize}
  \item The first byte of the message shall define which parameter is to
    be configured.
    \paragraph{\texttt{0x6d 0x6c `ml'}: Set MBus Full Prefix}
      \begin{itemize}
        \item[]
          \begin{bytefield} \\
            \colorbitbox{lightgreen}{8}{`m'} &
            \colorbitbox{lightred}{8}{Event ID} &
            \colorbitbox{lightcyan}{8}{Len (Must be 7)} &
            \colorbitbox{lightergreen}{8}{`l'} &
            \bitbox{8}{Ones Mask} &
            \bitbox{8}{Zeros Mask}
          \end{bytefield}
        \item {\bf Default:} {\tt 0xfffff0 0xfffff0} (Disabled)
        \item \hl{NOTE: This command is not supported by the current MBus implementation.}
        \item This shall be followed by 6 bytes. The first three bytes shall
          be considered the ones mask the second three bytes shall be
          considered the zeros mask. The masks are 20~bits long, the bottom
          4~bits of the transmitted masks shall be ignored.
      \end{itemize}
    \paragraph{\texttt{0x6d 0x73 `ms'}: Set MBus Short Prefix}
      \begin{itemize}
        \item[]
          \begin{bytefield} \\
            \colorbitbox{lightgreen}{8}{`m'} &
            \colorbitbox{lightred}{8}{Event ID} &
            \colorbitbox{lightcyan}{8}{Len (Must be 3)} &
            \colorbitbox{lightergreen}{8}{`s'} &
            \bitbox{8}{Ones Mask} &
            \bitbox{8}{Zeros Mask}
          \end{bytefield}
        \item {\bf Default:} {\tt 0xf0 0xf0} (Disabled)
        \item \hl{NOTE: This command is not supported by the current MBus implementation.}
        \item This shall be followed by 2 bytes. The first byte shall
          be considered the ones mask the second byte shall be
          considered the zeros mask. The masks are 4~bits long, the bottom
          4~bits of the transmitted masks shall be ignored.
        \item {\bf NOTE:} Changing the short prefix after enumeration is a
          violation of the MBus protocol. The ICE board will permit this, but
          it may have unexpected consequences.
      \end{itemize}
    \paragraph{\texttt{0x6d 0x4c `mL'}: Set MBus Snoop Full Prefix}
      \begin{itemize}
        \item[]
          \begin{bytefield} \\
            \colorbitbox{lightgreen}{8}{`m'} &
            \colorbitbox{lightred}{8}{Event ID} &
            \colorbitbox{lightcyan}{8}{Len (Must be 7)} &
            \colorbitbox{lightergreen}{8}{`L'} &
            \bitbox{8}{Ones Mask} &
            \bitbox{8}{Zeros Mask}
          \end{bytefield}
        \item {\bf Default:} {\tt 0xfffff0 0xfffff0} (Disabled)
        \item \hl{NOTE: This command is not supported by the current MBus implementation.}
        \item This shall be followed by 6 bytes. The first three bytes shall
          be considered the ones mask the second three bytes shall be
          considered the zeros mask. The masks are 20~bits long, the bottom
          4~bits of the transmitted masks shall be ignored.
        \item Messages matching the snoop prefix shall be reported by a
          {\tt `B'} message but will not be ACK'd on the physical bus by ICE.
        \item In the case of a conflict between a snoop and a match, the match
          takes precedence and ICE will ACK the message.
      \end{itemize}
    \paragraph{\texttt{0x6d 0x53 `mS'}: Set MBus Snoop Short Prefix}
      \begin{itemize}
        \item[]
          \begin{bytefield} \\
            \colorbitbox{lightgreen}{8}{`m'} &
            \colorbitbox{lightred}{8}{Event ID} &
            \colorbitbox{lightcyan}{8}{Len (Must be 3)} &
            \colorbitbox{lightergreen}{8}{`S'} &
            \bitbox{8}{Ones Mask} &
            \bitbox{8}{Zeros Mask}
          \end{bytefield}
        \item {\bf Default:} {\tt 0xf0 0xf0} (Disabled)
        \item \hl{NOTE: This command is not supported by the current MBus implementation.}
        \item This shall be followed by 2 bytes. The first byte shall
          be considered the ones mask the second byte shall be
          considered the zeros mask. The masks are 4~bits long, the bottom
          4~bits of the transmitted masks shall be ignored.
        \item Messages matching the snoop prefix shall be reported by a
          {\tt `B'} message but will not be ACK'd on the physical bus by ICE.
        \item In the case of a conflict between a snoop and a match, the match
          takes precedence and ICE will ACK the message.
      \end{itemize}
    \paragraph{\texttt{0x6d 0x73 `mb'}: Set MBus Broadcast Mask}
      \begin{itemize}
        \item[]
          \begin{bytefield} \\
            \colorbitbox{lightgreen}{8}{`m'} &
            \colorbitbox{lightred}{8}{Event ID} &
            \colorbitbox{lightcyan}{8}{Len (Must be 1)} &
            \colorbitbox{lightergreen}{8}{`b'} &
            \bitbox{8}{Ones Mask} &
            \bitbox{8}{Zeros Mask}
          \end{bytefield}
        \item {\bf Default:} {\tt 0x0f 0x0f} (Disabled)
        \item \hl{NOTE: This command is not supported by the current MBus implementation.}
        \item This shall be followed by 2 bytes. The first byte shall
          be considered the ones mask the second byte shall be
          considered the zeros mask. The masks are 4~bits long, the top
          4~bits of the transmitted masks shall be ignored.
        \item These masks are the set of broadcast channels that the ICE board
          should ACK on the physical MBus.
      \end{itemize}
    \paragraph{\texttt{0x6d 0x53 `mB'}: Set MBus Snoop Broadcast Mask}
      \begin{itemize}
        \item[]
          \begin{bytefield} \\
            \colorbitbox{lightgreen}{8}{`m'} &
            \colorbitbox{lightred}{8}{Event ID} &
            \colorbitbox{lightcyan}{8}{Len (Must be 1)} &
            \colorbitbox{lightergreen}{8}{`B'} &
            \bitbox{8}{Ones Mask} &
            \bitbox{8}{Zeros Mask}
          \end{bytefield}
        \item {\bf Default:} {\tt 0x0f 0x0f} (Disabled)
        \item \hl{NOTE: This command is not supported by the current MBus implementation.}
        \item This shall be followed by 2 bytes. The first byte shall
          be considered the ones mask the second byte shall be
          considered the zeros mask. The masks are 4~bits long, the top
          4~bits of the transmitted masks shall be ignored.
        \item These masks are the set of broadcast channels that the ICE board
          will not ACK on the physical MBus but will report as a snooped
          message (a {\tt `B'} message).
        \item In the case of a conflict between a snoop and a match, the match
          takes precedence and ICE will ACK the message.
      \end{itemize}
    \paragraph{\texttt{0x6d 0x6d `mm'}: Set master mode on/off}
      \begin{itemize}
        \item[]
          \begin{bytefield} \\
            \colorbitbox{lightgreen}{8}{`m'} &
            \colorbitbox{lightred}{8}{Event ID} &
            \colorbitbox{lightcyan}{8}{Len (Must be 2)} &
            \colorbitbox{lightergreen}{8}{`m'} &
            \bitbox{8}{\texttt{0x00} or \texttt{0x01}}
          \end{bytefield}
        \item {\bf Default:} {\tt 0x0} (Off)
        \item Boolean whether ICE should act as the MBus master node (1 - on)
          or as a MBus member node (0 - off).
      \end{itemize}
    \paragraph{\texttt{0x6d 0x63 `mc'}: Clock Speed}
      \begin{itemize}
        \item[]
          \begin{bytefield} \\
            \colorbitbox{lightgreen}{8}{`o'} &
            \colorbitbox{lightred}{8}{Event ID} &
            \colorbitbox{lightcyan}{8}{Len (4)} &
            \colorbitbox{lightergreen}{8}{`c'} &
            \bitbox{8}{Clock Divider}
          \end{bytefield}
        \item {\bf Default:} {\tt 0x000020} (10~MHz / 0x000020 = 312.5~kHz)
        \item This shall be followed by a three byte value N (MSB-first), where 10~MHz / N yields the desired clock speed.
        \item This configuration is meaningful only if ICE is configured as
          the MBus master node (see {\tt `mm'}).
      \end{itemize}
    \paragraph{\texttt{0x6d 0x69 `mi'}: Set should interrupt}
      \begin{itemize}
        \item[]
          \begin{bytefield} \\
            \colorbitbox{lightgreen}{8}{`m'} &
            \colorbitbox{lightred}{8}{Event ID} &
            \colorbitbox{lightcyan}{8}{Len (Must be 2)} &
            \colorbitbox{lightergreen}{8}{`i'} &
            \bitbox{8}{{\tt SHOULD\_INT}}
          \end{bytefield}
        \item {\bf Default:} {\tt 0x0} (Off)
        \item \hl{NOTE: This command is not supported by the current MBus implementation.}
        \item Configures whether ICE should interrupt the bus to transmit its
          next message. Possible values:
          \begin{itemize}
            \item[0] Do not interrupt
            \item[1] Interrupt (if necessary) for next message only. That is,
              this value is reset to 0 after the next {\tt `b'} command sent
              to ICE is handled.
            \item[2] Interrupt for all messages
          \end{itemize}
      \end{itemize}
    \paragraph{\texttt{0x6d 0x70 `mp'}: Set should use priority arbitration}
      \begin{itemize}
        \item[]
          \begin{bytefield} \\
            \colorbitbox{lightgreen}{8}{`m'} &
            \colorbitbox{lightred}{8}{Event ID} &
            \colorbitbox{lightcyan}{8}{Len (Must be 2)} &
            \colorbitbox{lightergreen}{8}{`p'} &
            \bitbox{8}{{\tt SHOULD\_PRIO}}
          \end{bytefield}
        \item {\bf Default:} {\tt 0x0} (Off)
        \item Configures whether ICE should send high priority MBus messages. Possible values:
          \begin{itemize}
            \item[0] Do not send priority message.
            \item[1] Send all messages as high priority.
          \end{itemize}
      \end{itemize}
  \paragraph{\texttt{0x6d} Responses}
    \begin{itemize}
      \item NAKs for this message shall be composed of an error
        code, followed by an optional explanitory string.
        \begin{itemize}
          \item[]
            \begin{bytefield} \\
              \colorbitbox{lightgreen}{8}{NAK (0x01)} &
              \colorbitbox{lightred}{8}{Event ID} &
              \colorbitbox{lightcyan}{8}{Len (Min: 1)} &
              \colorbitbox{lightblue}{8}{EINVAL (0x16)} &
              \bitbox{8}{[{\tt "Out of Range"}]} &
            \end{bytefield}
          \item {\texttt {\textbf EINVAL (22,0x16):}} Invalid argument.
          \item[]
            \begin{bytefield} \\
              \colorbitbox{lightgreen}{8}{NAK (0x01)} &
              \colorbitbox{lightred}{8}{Event ID} &
              \colorbitbox{lightcyan}{8}{Len (Min: 1)} &
              \colorbitbox{lightblue}{8}{ENODEV (0x13)} &
            \end{bytefield}
          \item {\texttt {\textbf ENODEV (19,0x13):}} The
            implementation does not support changing or querying this parameter. Unless
            otherwise specified, it {\bf MUST} be hardcoded to the default.
        \end{itemize}
    \end{itemize}
\end{itemize}

\subsubsection{\texttt{0x47 `G'} -- Query GPIO State / Configuration}
{\em Synchronous Request}
\begin{itemize}
  \item[]
    \begin{bytefield} \\
      \colorbitbox{lightgreen}{8}{`G' (0x47)} &
      \colorbitbox{lightred}{8}{Event ID} &
      \colorbitbox{lightcyan}{8}{Len (Must be 1)} &
      \bitbox{8}{GPIO IDX}
    \end{bytefield}
  \item These messages complement the set GPIO (`g') messages.
  \item An ACK response should mimic the corresponding set message.
\end{itemize}

\subsubsection{\texttt{0x67 `g'} -- Set / Configure GPIO}
{\em Synchronous Request, Asynchronous Message}
\begin{itemize}
  \item The first byte of this message shall be a specificer, indicating
    what type of GPIO action is requested.
  \paragraph{\texttt{0x67 0x6c `gl'} -- GPIO Level}
    \begin{itemize}
      \item[]
        \begin{bytefield} \\
          \colorbitbox{lightgreen}{8}{`o' (0x67)} &
          \colorbitbox{lightred}{8}{Event ID} &
          \colorbitbox{lightcyan}{8}{Len (Must be 4)} &
          \colorbitbox{lightergreen}{8}{`l' (0x6c)} &
          \bitbox{8}{GPIO Mask}
        \end{bytefield}
      \item {\bf Default:} {\tt 0x000000} (All low)
      \item This shall be followed by a three byte bitmask indicating the desired output level of all 24 GPIO.  This is only reflected in those GPIO which are configured as output. Each bit shall be valued:
      \begin{itemize}
        \item 0: Low (0.0 V)
        \item 1: High (1.2 V)
      \end{itemize}
    \end{itemize}
  \paragraph{\texttt{0x67 0x64 `gd'} -- GPIO Direction}
    \begin{itemize}
      \item[]
        \begin{bytefield} \\
          \colorbitbox{lightgreen}{8}{`o' (0x67)} &
          \colorbitbox{lightred}{8}{Event ID} &
          \colorbitbox{lightcyan}{8}{Len (Must be 4))} &
          \colorbitbox{lightergreen}{8}{`d' (0x64)} &
          \bitbox{8}{GPIO Mask}
        \end{bytefield}
      \item {\bf Default:} {\tt 0x000000} (All input)
      \item This shall be followed by a three byte bitmask indicating the desired direction of all 24 GPIO.  The output level of each GPIO which are configured as output are set via the `Set GPIO Level' ({\tt `gl'}) command.  Each bit shall be valued:
        \begin{itemize}
          \item 0: Input (DEFAULT)
          \item 1: Output
        \end{itemize}
    \end{itemize}
  \paragraph{\texttt{0x67 0x69 `gi'} -- GPIO Interrupt Enable}
    \begin{itemize}
      \item[]
        \begin{bytefield} \\
          \colorbitbox{lightgreen}{8}{`o' (0x67)} &
          \colorbitbox{lightred}{8}{Event ID} &
          \colorbitbox{lightcyan}{8}{Len (Must be 4))} &
          \colorbitbox{lightergreen}{8}{`d' (0x64)} &
          \bitbox{8}{GPIO Mask}
        \end{bytefield}
      \item {\bf Default:} {\tt 0x000000} (All interrupts disabled)
      \item This shall be followed by a three byte bitmask indicating whether a change in each GPIO bit wil generate an GPIO level change message.  Only those GPIO which are configured as input will generate interrupts.  Each bit shall be valued:
        \begin{itemize}
          \item 0: Interrupt Disabled (DEFAULT)
          \item 1: Interrupt Enabled
        \end{itemize}
    \end{itemize}
%  \paragraph{\texttt{0x67} Responses}
%    \begin{itemize}
%      \item {\texttt {\textbf ENODEV (19,0x13):}} The requested
%        GPIO does not exist.
%      \item[]
%        \begin{bytefield} \\
%          \colorbitbox{lightgreen}{8}{NAK (0x01)} &
%          \colorbitbox{lightred}{8}{Event ID} &
%          \colorbitbox{lightcyan}{8}{Len (Min: 1)} &
%          \colorbitbox{lightblue}{8}{ENODEV (0x13)} &
%          \bitbox{8}{[{\tt "No such GPIO"}]} &
%        \end{bytefield}
%      \item {\texttt {\textbf EINVAL (22,0x16):}} The requested GPIO
%        exists, but cannot be configured this way at this time.
%      \item[]
%        \begin{bytefield} \\
%          \colorbitbox{lightgreen}{8}{NAK (0x01)} &
%          \colorbitbox{lightred}{8}{Event ID} &
%          \colorbitbox{lightcyan}{8}{Len (Min: 1)} &
%          \colorbitbox{lightblue}{8}{EINVAL (0x16)} &
%          \bitbox{8}{[{\tt "GPIO is input"}]} &
%        \end{bytefield}
%    \end{itemize}
\end{itemize}

\subsubsection{\texttt{0x50 `P'} -- Query Power State}
{\em Synchronous Request}
\begin{itemize}
  \item[]
    \begin{bytefield} \\
      \colorbitbox{lightgreen}{8}{`P' (0x50)} &
      \colorbitbox{lightred}{8}{Event ID} &
      \colorbitbox{lightcyan}{8}{Len (Must be 1)} &
      \colorbitbox{lightergreen}{8}{Parameter} &
      \bitbox{8}{PWR IDX}
    \end{bytefield}
  \item These messages complement the Set Power (`p') messages.
  \item The `Parameter' field is the parameter specifier to query.
  \item An ACK response should mimic the corresponding set message.
\end{itemize}

\subsubsection{\texttt{0x70 `p'} -- Set Power State}
{\em Synchronous Request, Asynchronous Message}
\begin{itemize}
  \item Set Power State messages allow direct control of set-point voltage and on/off states for various power domains on the ICE board.  The first byte of this message shall be a specificer, indicating which parameter is requested.  The second byte of the message shall be the power domain identifier.  Currently implemented power domain identifiers are:
   \begin{itemize}
     \item 0: M3 0.6V (0.675V Default)
     \item 1: M3 1.2V (1.2V Default)
     \item 2: M3 VBatt (3.8V Default)
   \end{itemize}
  \paragraph{\texttt{0x70 0x76 `pv'} -- Voltage State}
    \begin{itemize}
      \item The first byte of the message shall be a single byte indicating the power domain identifier to set.  The second byte ($v\_set$) shall specify the voltage according to the equation:
$$V_{out} = (0.537 + 0.0185*v\_set)*V_{default}$$
      Valid values for $v\_set$ range from {\tt 0} to {\tt 31}
      \item[]
        \begin{bytefield} \\
          \colorbitbox{lightgreen}{8}{`p' (0x70)} &
          \colorbitbox{lightred}{8}{Event ID} &
          \colorbitbox{lightcyan}{8}{Len (Must be 3)} &
          \colorbitbox{lightergreen}{8}{`v' (0x76)} &
          \bitbox{8}{PWR IDX} &
          \bitbox{8}{$v\_set$}
        \end{bytefield}
    \end{itemize}
  \paragraph{\texttt{0x70 0x6f `po'} -- On/Off State}
    \begin{itemize}
      \item The first byte of the message shall be a single byte indicating the power domain identifier to set.  The second byte shall be
        valued {\tt 0} or {\tt 1}, depending on the desired On/Off
        state.
      \item[]
        \begin{bytefield} \\
          \colorbitbox{lightgreen}{8}{`p' (0x70)} &
          \colorbitbox{lightred}{8}{Event ID} &
          \colorbitbox{lightcyan}{8}{Len (Must be 3)} &
          \colorbitbox{lightergreen}{8}{`o' (0x6f)} &
          \bitbox{8}{PWR IDX} &
          \bitbox{8}{On/Off}
        \end{bytefield}
    \end{itemize}
\end{itemize}
